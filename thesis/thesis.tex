%
% thesis.tex
%
% Master's Thesis/Ph.D. Dissertation Template
% Clemson University
%

%
% The document guidelines say the font can be between 10pt and 12pt.
% Specify whatever you want it to be here.
%
\documentclass[10pt]{ClemsonThesis}

%
% Use any additional packages you might need
%
%% \usepackage{listings}
%% \usepackage{comment}
\usepackage{amsmath,bm}
\usepackage{minted}
\usepackage{rotating}
\usepackage{multirow}

%
% Make the document your own -- fill in these values to reflect the type of
% document you are writing.
%
\title{Evaluation of compact finite differences on GPU-accelerated clusters}
\department{Mechanical Engineering}
\documentType{Master's Thesis}
\major{Mechanical Engineering}
\degree{Master of Science}
\graduationMonth{Dec}
\graduationYear{2015}
\author{Ashwin Srinath}
\committeeChair{Dr. Richard S. Miller}
\committeeMemberOne{Dr. Gang Li}
\committeeMemberTwo{Dr. Lonny L. Thompson}
%% optional (for Master's) \committeeMemberThree{Dr. John Doe}
%% optional \committeeMemberFour{Dr. Jane Doe}
%% optional \committeeMemberFive{Dr. Mary Doe}
%% optional \committeeMemberSix{Dr. Mark Doe}

%
% PDF Setup -- most of this you do not need to touch
%
\hypersetup{
    colorlinks,
    linkcolor={black},
    citecolor={black},
    filecolor={black},
    urlcolor={black},
    pdftitle={\theTitle},
    pdfauthor={\theAuthor},
    pdfsubject={\theDocumentType},
    pdfkeywords={Clemson University, \theDepartment, \theDocumentType, \theMajor, \theDegree},
    pdfstartpage={1},
}
%
% User-specified command definitions/redefinitions
%
\newcommand{\cplusplus}{{\rm C\raise.5ex\hbox{\small ++}}}
%%\newcommand{\num}[1]{\mbox{(\textit{#1})}}
%%\renewcommand{\ttdefault}{pcr}
%%\renewcommand\lstlistlistingname{List of Listings}

\begin{document}
%  ============================================================================
    \frontmatter % Begin front matter (pages are numbered with Roman numerals)
%  ============================================================================

    \addtotoc{Title Page}{\maketitle}          % Generate the title page
    \doublespacing                             % Text should be double spaced
    \setcounter{page}{2}                       % Abstract begins on page 2
    \setcounter{tocdepth}{2}
    \setcounter{secnumdepth}{2}
    \addtotoc{Abstract}{\chapter*{Abstract}

This is the abstract
}  % Generate the abstract
    %
    % The dedication page is optional.  Comment out this line if you do not
    % want to include this page.
    %
    \addtotoc{Dedication}{\chapter*{Dedication}

I dedicate this work to my parents,
and to my brother, Akhil,
who pushes me always to be a better example.
}
    %
    % The acknowledgment page is optional.  Comment out this line if you do
    % not want to include this page.
    %
    \addtotoc{Acknowledgments}{\chapter*{Acknowledgments}

I owe my deepest thanks
to my advisor, Dr. Richard S. Miller,
for his immense guidance and support;
and the members of my committee,
Dr. Gang Li and Dr. Lonny Thompson
for their valuable time.
I would like to acknowledge
Dr. Daniel Livescu at Los Alamos National Lab
for introducing us to this interesting problem,
for his helpful comments,
and for generous funding.
I also thank
Dr. Melissa Smith and Karan Sapra
for their helpful guidance,
and for organizing an excellent course on GPU computing.
I owe a great deal of gratitude to
Dr. Christopher Cox at Clemson University
for his guidance on a range of topics
and for many interesting conversations.
I would like to thank the
staff of the Cyberinfrastructure Technology Integration group at Clemson,
and especially
Dr. Galen Collier and Dr. Marcin Ziolkowski
for their valuable comments and support.
I owe thanks to Clemson University for the generous
allocation of compute time on the Palmetto cluster.
I acknowledge
Dr. K. N. Seetharamu and Dr. V. Krishna,
my advisors at PES Institute of Technology
for inspiring in me a passion for
scientific research.
I am indebted to Dr. Anush Krishnan
for introducing me to the fascinating field of
scientific computing.
I owe a special thanks to the
Software Carpentry Foundation
and its members,
for helping me be
a more productive programmer,
a more deliberate learner,
a more effective teacher,
and a better person.

Finally, I thank my family and friends
for everything they have done for me.
}
    \singlespacing                             % Single space the lists
    \tableofcontents \clearpage                % Generate the Table of Contents

    %
    % REMEMBER: Review your caption listings in the genrated lists
    %           and make sure they include '\newline' commands as necessary.
    %           See the README for further information.
    %
    \addtotoc{List of Tables}{\listoftables}   % Generate the List of Tables
    \addtotoc{List of Figures}{\listoffigures} % Generate the List of Figures

    %
    % Include other optional lists.  Computer science, for example, would
    % likely include a 'List of Listings' (and would \usepackage{listings}
    % and \renewcommand\lstlistlistingname{List of Listings}).
    %
    %% \addtotoc{List of Listings}{\lstlistoflistings}


%  ===========================================================================
    \mainmatter % Begin main matter (pages are numbered with Arabic numerals)
%  ===========================================================================
    \doublespacing % Text should be double spaced

    %
    % Here we have each chapter in a separate file.  Name these as you choose,
    % and include them in the order you want them to appear.  Be sure to use
    % the \inputfile command.
    %
    \inputfile{introduction-and-background.tex}
    \inputfile{proposed-tridiagonal-algorithm.tex}
    \inputfile{compact-finite-difference.tex}
    \inputfile{results.tex}
    \inputfile{conclusions.tex}

    %
    % The appendices are optional.  This is the format for two or more.
    % If you do not wish to include an appendix, comment out these lines.
    % If you want just one, see the formatting guidelines.
    %
    %\begin{appendices}
    %    \begin{subappendices}
    %        \inputfile{appendixA.tex}
    %        \inputfile{appendixB.tex}
    %        \inputfile{appendixC.tex}
    %    \end{subappendices}
    %\end{appendices}

    \singlespacing                             % Single space the Bibliography
    %
    % The bibliography style.  Set this to whatever matches you discipline.
    % For example, Computer Science would likely use 'plain'.  You might
    % also want to change the name from 'Bibliography' to 'References'
    % or 'Work Cited'.
    %
    % 'plain'   gets you numbered references and citations (e.g., [1] Dyson).
    %
    % 'alpha'   gets you labels formed from an abbreviation of the authors'
    %           names and the year of publication.  If there is more than
    %           one author, it will use the first letter of up to the first
    %           three authors' last names.
    %
    %           Some examples:
    %               [DED01] F.W. Dyson, A.G. Edgar, and D.B. Denny ... 2001
    %               [DE01] F.W. Dyson, A.G. Edgar ... 2001
    %               [Dys01] F.W. Dyson ... 2001
    %
    % 'apalike' gets you labels formed from the authors' names and year of
    %           publication.
    %
    %           Some examples:
    %               [Dyson et al., 2001] F.W. Dyson, A.G. Edgar, and
    %                 D.B. Denny ... 2001
    %               [Dyson and Edgar, 2001] F.W. Dyson, A.G. Edgar ... 2001
    %               [Dyson, 2001] F.W. Dyson ... 2001
    %
    \nocite{*}
    \bibliographystyle{plain}
    \addtotoc{Bibliography}{\bibliography{bibliography}}
\end{document}
