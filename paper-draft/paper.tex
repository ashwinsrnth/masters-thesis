\documentclass{elsarticle}

\usepackage{amsmath}


\begin{document}

\begin{frontmatter}
\author{Ashwin Srinath}
\title{A}
\maketitle

\begin{abstract}
    We present a computation strategy for evaluating
    compact finite differences efficiently on GPU clusters.
    We describe a novel algorithm for solving the
    near-Toeplitz tridiagonal systems associated with
    the compact finite difference schemes.
    Our specialized tridiagonal solver
    takes advantage of the near-Toeplitz nature of the
    tridiagonal system to precompute the coefficients
    appearing in the Cyclic Reduction algorithm.
    This allows us to solve the tridiagonal systems
    up to 2 times faster than with the NVIDIA CUSPARSE
    gtsvStridedBatch routine.
    Additionally, we present a methodology to solve for
    compact finite differences fully on multiple GPUs on a cluster
    without intermediate host-device transfers.
\end{abstract}

\end{frontmatter}
    
\section{Introduction}
      
\section{Compact Finite Difference Schemes}

For example,
in a uniformly spaced one-dimensional grid with spacing $dx$,
if $f_i$ represents the value of
the function evaluated at the $i$th node,
the first derivative $f^{\prime}_i$ can be approximated from
a relation of the form:

\begin{equation}
\begin{split}
    \beta(f^{\prime}_{i-2} + f^{\prime}_{i+2}) + \
    \alpha(f^{\prime}_{i-1} + f^{\prime}_{i+1}) + \
        f^{\prime}_i
    = 
    c\frac{f_{i+3} - f_{i-3}}{dx} + \
    b\frac{f_{i+2} - f_{i-2}}{dx} + \\
    a\frac{f_{i+1} - f_{i-1}}{dx} + \
    \hdots
\end{split}
\label{eqn:general-compact}
\end{equation}

The derivatives near the boundaries are approximated using
one-sided stencil operators of the form:

\begin{equation}
f^{\prime}_1 + \alpha_1 f^{\prime}_2 = \
    \frac{1}{dx} (a_1 f_1 + b_1 f_2 + c_1 f_3 + d_1 f_4) 
\end{equation}

and 

\begin{equation}
    f^{\prime}_n + \alpha_n f^{\prime}_{n-1} = \
    \frac{1}{dx} (a_n f_n + b_n f_{n-1} + c_n f_{n-2} + d_n f_{n-3}) 
\end{equation}

As described by Lele \cite{lele1992compact},
setting $\beta = 0$ in Equation \ref{eqn:general-compact} leads to
a family of tridiagonal schemes with a single parameter ($\alpha$).
The coefficient matrix associated with such tridiagonal schemes
has the general form:

\begin{equation} \label{eqn:compact-tridiagonal-system}
\begin{bmatrix}
     1 &  \alpha_1  \\
     \alpha   &  1   &  \alpha \\
         &  \alpha   &  1  &  \alpha  \\
         &      &  \alpha  &  1  &  \alpha  \\
         &      &     &     &  \ddots \\
         &      &     &     &     &  \ddots  \\
         &      &     &     &     &  \alpha_n &  1
\end{bmatrix}
\begin{bmatrix}
    f^{\prime}_1 \\
    f^{\prime}_2 \\
    f^{\prime}_3 \\
    \vdots \\
    \vdots \\
    f^{\prime}_{n-1} \\
    f^{\prime}_n
 \end{bmatrix}
=
\begin{bmatrix}
   d_1 \\
   d_2 \\
   d_3 \\
   \vdots \\
   \vdots \\
   d_{n-1} \\
   d_{n}
\end{bmatrix}
\end{equation}

We note that the system is \emph{near Toeplitz tridiagonal},
i.e., the left-hand-side matrix is tridiagonal with
constant coefficients along each diagonal,
except for the first and last equations.

\section{Distributed compact-finite difference solution} 

We use the algorithm discussed by Mattor et al.
\cite{mattor1995algorithm}
to distribute and solve the problem among multiple processes. 
We present the algorithm to solve a general tridiagonal system,
and in subsequent sections discuss its adaptation to solving
near-Teoplitz tridiagonal systems on GPUs.

Given a tridiagonal system with  $n$ equations,
to be solved by $P$ processes:

\begin{align}
& \begin{bmatrix}
b_1^p & c_1^p \\
a_2^p & b_2^p & c_2^p \\
      & a_3^p & b_3^p & c_3^p \\
      &       & a_4^p & b_4^p & c_4^p \\
      &       &       &       &  \ddots & c_{m-1}^p\\
      &       &       &       &     a_{m}^p  & b_{m}^p
\end{bmatrix}
\begin{bmatrix}
x_{r,1}^p \\
x_{r,2}^p \\
x_{r,3}^p \\
x_{r,4}^p \\
\vdots \\
x_{r,m}^p
\end{bmatrix}
=
\begin{bmatrix}
r_1^p \\
r_2^p \\
r_3^p \\
r_4^p \\
\vdots \\
r_m^p
\end{bmatrix} & \label{eqn:global-system} 
\end{align}

each process $p$ proceeds by solving the following three ``local''
tridiagonal systems:

\begin{align}
& \begin{bmatrix}
b_1^p & c_1^p \\
a_2^p & b_2^p & c_2^p \\
      & a_3^p & b_3^p & c_3^p \\
      &       & a_4^p & b_4^p & c_4^p \\
      &       &       &       &  \ddots & c_{m-1}^p\\
      &       &       &       &     a_{m}^p  & b_{m}^p
\end{bmatrix}
\begin{bmatrix}
x_{r,1}^p \\
x_{r,2}^p \\
x_{r,3}^p \\
x_{r,4}^p \\
\vdots \\
x_{r,m}^p
\end{bmatrix}
=
\begin{bmatrix}
r_1^p \\
r_2^p \\
r_3^p \\
r_4^p \\
\vdots \\
r_m^p
\end{bmatrix} & \label{eqn:primary-system} \\
%
%
%
& \begin{bmatrix}
b_1^p & c_1^p \\
a_2^p & b_2^p & c_2^p \\
      & a_3^p & b_3^p & c_3^p \\
      &       & a_4^p & b_4^p & c_4^p \\
      &       &       &       &  \ddots & c_{m-1}^p\\
      &       &       &       &     a_{m}^p  & b_{m}^p
\end{bmatrix}
\begin{bmatrix}
u_1^p \\
u_2^p \\
u_3^p \\
u_4^p \\
\vdots \\
u_m^p
\end{bmatrix}
=
\begin{bmatrix}
-a_1^p \\
0 \\
0 \\
0 \\
\vdots \\
0
\end{bmatrix} & \label{eqn:secondary-system-1} \\
%
%
%
& \begin{bmatrix}
b_1^p & c_1^p \\
a_2^p & b_2^p & c_2^p \\
      & a_3^p & b_3^p & c_3^p \\
      &       & a_4^p & b_4^p & c_4^p \\
      &       &       &       &  \ddots & c_{m-1}^p\\
      &       &       &       &     a_{m}^p  & b_{m}^p
\end{bmatrix}
\begin{bmatrix}
l_1^p \\
l_2^p \\
l_3^p \\
l_4^p \\
\vdots \\
l_m^p
\end{bmatrix}
=
\begin{bmatrix}
0 \\
0 \\
0 \\
0 \\
\vdots \\
-c_m^p
\end{bmatrix} & \label{eqn:secondary-system-2}
\end{align}

We call the system in Equation \ref{eqn:primary-system}
the ``primary'' system, and the systems in
Equations
\ref{eqn:secondary-system-1} and \ref{eqn:secondary-system-2}
the ``secondary'' systems.
The local part of the solution to the ``global'' tridiagonal system
(Equation \ref{eqn:global-system})
is obtained as a linear combination of
the solutions to the primary and secondary sytems:

\begin{equation}
    \boldsymbol{x}^p = \boldsymbol{x}_r^p + \
        \alpha^p \boldsymbol{u}^p + \beta^p \boldsymbol{l}^p
    \label{eqn:sum-of-systems}
\end{equation}

where the  parameters $\alpha^p$ and $\beta^p$ are obtained by
solving the following ``reduced'' system of equations:

\begin{equation} \label{eqn:reduced-system}
\begin{bmatrix}
l^1_m & -1 \\
-1    & u^2_1 & l^2_1 \\
      & u^2_m & l^2_m & -1 \\
      &       & -1    & u^3_1 & l^3_1 \\
      &       &       & u^3_m & l^3_n  & -1 \\
      &       &       &       & \ddots & \ddots & \ddots \\
      &       &       &       &        & -1     & u^P_1
\end{bmatrix}
\begin{bmatrix}
\beta^1 \\
\alpha^2 \\
\beta^2 \\
\alpha^3 \\
\beta^3 \\
\vdots \\
\alpha^P
\end{bmatrix}
=
\begin{bmatrix}
x_{r,m}^1 \\
x_{r,1}^2 \\
x_{r,m}^2 \\
x_{r,1}^3 \\
x_{r,m}^3 \\
\vdots \\
x_{r,1}^P \\
\end{bmatrix}
\end{equation}

$P = n/m$ is the number of processes.
We note that the reduced system is sized $2P-2$,
and that its assembly requires communication between the processes.
The general solution procedure is then:

\begin{enumerate}
    \item Each process $p$ solves the primary and secondary systems
        to obtain $\boldsymbol{x}_r^p$, $u^p$ and $l^p$
    \item The ``reduced'' system is assembled and solved for
        the coefficients $\alpha^p$ and $\beta^p$
    \item The local solution is computed from Equation \ref{eqn:sum-of-systems}
\end{enumerate}


\subsection{Evaluation of the right-hand side}

\subsection{Evaluation of the right-hand side}

\subsection{Solution of primary Near-Toeplitz tridiagonal systems}

\subsection{Solution of reduced system}

\subsection{Solution of secondary systems}

\section{Results}


\section*{References}

\bibliography{references}
\bibliographystyle{elsarticle-num}
\end{document}
